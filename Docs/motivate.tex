\section{Motivation}
As stated in Introduction, there are several real-world settings where applications 
require communication between two or more RSMs. 
One such setting is the cross-chain communication in the decentralized cryptocurrency ecosystem.
Cross-chain communication faciltates transfer of tokens (a form of digital currency) from 
one RSM to another. 
For instance, Cosmos aims to allow its users exchange their Bitcoins for equivalent amount of Ethers.
To do so, Cosmos employs the {\em Interblockchain communication protocol} (\IBC{})~\cite{ibc}.
\IBC{} makes no assumption on the architecture of communicating RSMs and the data communicated.
However, \IBC{} requires each RSM to order transactions through a \BFT{} consensus and 
expects commitment proofs for each data.

\IBC{} aims to act as an interface between modules (such as smart contracts or other ledger components), 
consensus protocols, blockchains, and so on.
To communicate messages between two modules, \IBC{} requires a set of {\em relayer} processe.
These relayer processes continuously monitor the modules and exchange committed data on one ledger to another.
As relayer processes may fail, \IBC{} expects at least one correct relayer process to be always live.
To prevent data loss from byzantine relayer processes, \IBC{} adds a sequence number to each transmitted packet. 

