\section{Introduction}
\begin{itemize}
\item This paper presents \Shadow{}, the first message communication protocol 
between two Byzantine fault-tolerant (\BFT{}) replicated state machines (\RSM{}). 
\Shadow{} learns from the TCP-protocol and guarantees reliable and efficient 
communication even in the presence of adversaries.

\item On the one hand, \Shadow{} establishes efficient bi-directional communication channels 
for untrusted organizations that wish to reliably exchange data. 
On the other hand, \Shadow{} permits cross-chain communication among modern blockchain-based cryptocurrencies, 
such as Bitcoin, Ethereum, and Algorand.

\item Internally, each of these organizations and/or cryptocurrencies can 
be observed as an \RSM{} sharing data among multiple sub-organizations or participants, and 
run crash fault-tolerant (\CFT{}) consensus protocols for availability and 
byzantine fault-tolerant protocols for an additional data-integrity.

\item The concern for data-integrity within an organization arises 
from lack of trust among sub-organizations or participants. 
For example, some participants could act malicious.
Similarly, in the case of an incident, no sub-organization may take responsibility. 


%\item These untrusted parties within an organization may individually store a copy of all the shared data, 
%and any transaction that modifies the state of the shared data requires a consensus. 
%Hence, we observe each organization as a replicated state machine (RSM).

\item Communication between two or more \RSM{s} is even more challenging:
communicating \RSM{s} may want to share a subset of their data and
need to determine reliable communication channels. 

\item A simple way to solve this challenge is to require each \RSM{} to designate 
some parties as {\em trusted} and route all the communication through these 
trusted parties.
However, such trusted parties can be compromised and can face targeted attacks. 


\item As a result, recent works that expect communication between two or more \RSM{s} 
do not employ trusted parties and assume their nodes will {\em somehow} manage communication. 
Specifically, these works lack details regarding {\em how} this communication should take place 
and only focus on:
(i) the merits of communication between \RSM{s}, and 
(ii) the safe application of data communicated across \RSM{s}. 
The underlying assumption is that some protocol will guarantee {\em reliable and efficient communication} 
among the \RSM{s}.

\item Our protocol \Shadow{} fills this gap by answering following questions.
(1) How many senders and receivers are needed to reliably communicate one message? 
(2) How many copies of a message should each sender send?
(3) How frequently do the senders have to send the copies of the message?
(4) How does the sender know that the message has been received?
(5) Can the communication channel be full-duplex?

\item \Shadow{} provides communicating \RSM{s} access to full-duplex channels.
\Shadow{} permits sender to buffer messages until they are received and 
resend undelivered messages.
Further, \Shadow{} conserves channel bandwidth by allowing receivers to send cumulative acknowledgments.

\item Two key innovative takeaways of \Shadow{} are:
(1) Quorum acknowledgment (\quack{}), which provide the guarantee that the message has been received by 
at least one honest party at the receiving RSM.
(2) implicit \duck{}, which informs the sending RSM about a missing message. 
Further, these \quack{s} and \duck{s} are received asynchronously by RSMs, and 
do not bottleneck the ongoing stream of messages.
 
\item In the good case, \Shadow{} guarantees that each message is only communicated once by the 
sending RSM and a single cumulative acknowledgment from the receiver guarantees receipt of 
multiple messages.
Moreover, as \Shadow{} supports duplex communication, these cumulative acknowledgments can be piggybacked  
with the outgoing messages.

\item We argue that \Shadow{} provides optimal asynchronous duplex communication without the 
need for any advanced cryptographic primitives or network coding.
\Shadow{} is designed for permitting high throughput communication between the RSMs, while ensuring a bounded latency.

\end{itemize}
%A large number of existing applications require communication between multiple clusters of replicas. 
%Examples of such applications include: 
%(i) Sharded systems where cross-shard transactions need to access data from two or more independent shards.
%(ii) Geo-replicated systems where the data is replicated across clusters, which are spread across the globe;
%these clusters need to periodically exchange data to ensure consistency.
%(iii) Two or more \Name{SMR}s running distinct consensus protocols; recent cross-chain blockchain applications
%target connecting two different blockchains and permit exchange of different crypto-tokens.
%
%Each of these applications can be viewed as a collection of independent clusters where each cluster manages its 
%own data and needs data from other clusters.
%Assume that each cluster has $\n{}$ machines.
%A naive way to exchange a message (say $m$) between two clusters is to require every machine in the sending cluster to send the
%message $m$ to every machine in the receiving cluster. 
%But, such a communication policy will result in transmitting $\n{}^2$ copies of the message $m$.
%To optimize this, a recent work shows that it is sufficient to transmit only $\n{}$ copies of the message $m$ 
%to guarantee that every machine of the receiving cluster {\em eventually} receives the message.
%Although this solution is linear in the number of machines in a cluster, 
%its use in streaming applications is still prohibitive as these applications have to continuously transmit 
%messages and this solution requires sending $\n{}$ copies per message, which will easily bottleneck the network.
%
%To resolve these challenges, in this work, we present the design of a communication protocol that 
%in the best case requires only sending {\em one} copy per message. 
%We call our protocol as {\em Shadow} as we target it for applications where clusters participate in 
%bi-directional communication (each cluster will act as a sender for some message and a receiver for another) and 
%irrespective of their nature (sender or receiver), the protocol remains the same.
%



