\section{\Shadow{} Protocol}
\label{s:algo}





\begin{definition}\label{def:scrooge}
{\bf \Scrooge{}:}
An \RSM{}-to-\RSM{} asynchronous communication protocol that enables reliable message delivery with a constant 
number of message copies.
\end{definition}

\begin{definition}
{\bf Communication Abstraction:}
\Scrooge{} guarantees reliable communication of a message between two \RSM{s} if each \RSM{} includes at least $\n{} \ge 3\f{}+1$ 
replicas;
\Scrooge{} does not guarantee reliable communication between two replicas.
\end{definition}


\Scrooge{} provides the following \RSM{}-to-\RSM{} communication properties.
\begin{enumerate}[wide,nosep]
	\item \textbf{Basic Data Transfer:} \Scrooge{} provides a ``push'' functionality, which guarantees that any message, which a replica of the 
sending \RSM{} wants to send will be eventually sent to a receiver.
	\item \textbf{Message Reliability:} Any message dropped or delayed by Byzantine replicas is eventually received by honest replicas.
To achieve this property, we acknowledge receipt of each message.
	\item {\bf Flow Control:} \Scrooge{} ensures that a message is only re-send if and only if a sufficient number of honest replicas 
have implicitly acknowledged its non-receipt.
\end{enumerate}





To facilitate reliable \RSM{}-to-\RSM{} communication using the \Scrooge{} protocol, 
we require each replica to maintain several lists and maps, which
help to track the messages communicated across \RSM{s}.
Each message communicated using \Scrooge{} has two parts: 
(1) transaction part, which is the actual message committed at the \RSM{}, and 
(2) acknowledgment part, which is an acknowledgement for messages from the receiver \RSM{}. 
Let $\SSet{i}$ be the set of all messages to be sent by $i$-th \RSM{} $\SMR{i}$, such that:
\begin{equation*}
\SSet{i} = \{\SignMessage{m, 1}{\Qusign{i}},  \SignMessage{m, 2}{\Qusign{i}}, ... , \SignMessage{m, \Seqn}{\Qusign{i}}, ... \} 
\end{equation*}
%Here, $m', m''$ are committed by \RSM{} $\SMR{j}$, $j \neq i$. 
Each $l$-th replica $\Replica{i}{l}$ in \RSM{} $\SMR{i}$ maintains following:

\begin{itemize}[wide]
\item $\Sbuf{l} = \{\SignMessage{m, k}{\Qusign{i}} ~|~ k ~\equiv l \pmod{n_i} ~\wedge ~\SignMessage{m, k}{\Qusign{i}} \in \SSet{i} \}$, the set of messages to be sent $\Replica{i}{l}$.

\item $\Hbuf{l} = \SSet{i} - \Sbuf{l}$, the remaining set of messages.

\item $\Rbuf{l} = \{\SignMessage{m, k}{\Qusign{j}} ~|~ j \neq i ~\wedge ~\SignMessage{m, k}{\Qusign{j}} \in \SSet{j} \}$, the messages received from \RSM{} $\SMR{j}$.

\item $\SAck{l} = \{\Ack{k} ~|~ \SignMessage{m, k}{\Qusign{j}} \in \Rbuf{l} \}$, the acknowledgments for received messages.

\end{itemize}
%
Additionally, we assume that each replica in \RSM{} $\SMR{i}$ 
counts the number of acknowledgments it has received for 
each message in the set $\SSet{i}$ from the replicas in \RSM{} $\SMR{j}$.
It maintains the map $\RAck{l}{}$ for this purpose. 


{\bf Setup.}
Definition~\ref{def:scrooge} states that \Scrooge{} provides an \RSM{}-to-\RSM{} asynchronous communication.
This implies that \Scrooge{} supports {\em bi-directional} communication where both the \RSM{s} can send/receive 
messages at the same time.
This property allows us to conserve the bandwidth by piggybacking acknowledgments along with the messages.
For the rest of this section, we assume that \RSM{s} $\SMR{i}$ and $\SMR{j}$ want to communicate a 
steady stream of committed messages.
Further, \Scrooge{} requires each message that an \RSM{} plans to communicate has 
a monotonically increasing sequence number (say $\Seqn \ge 1$), and there are no gaps in these sequence numbers.


%Our \Shadow{} protocol expects that each replica in a cluster has a {\em sending buffer} (\Sbuf{}) and 
%a {\em receiving buffer} (\Rbuf{}) of finite sizes. 
%\Sbuf{} stores all the messages that a replica plans to send to the other cluster, while $\Rbuf{}$ 
%includes the messages from the other cluster. 
%We visualize $\Sbuf{}$ as a {\em queue} where new messages are added at the {\em rear} and 
%messages are removed from the {\em front}.
%Additionally, each message stored in $\Sbuf{}$ has an associated {\em counter}. 
%As the aim of our protocol is to reduce the number of messages communicated across the clusters, 
%so in the normal-case (failure-free) we require each replica to send a distinct message.
%We do this by dividing the messages to be sent between the replicas based on their sequence numbers ($k \mod \ID{} = 0$).
%Each replica can use this simple function to find out the messages it is expected to send and it stores them 
%in $\Sbuf{}$.
%The remaining messages (to be sent by other replicas) are stored in a {\em holding buffer} (\Hbuf{}) and 
%we also require a counter to be associated with each message in $\Hbuf{}$. 

\begin{example}\label{ex:rep-send}
Assume cluster $\Cluster{X}$ has $4$ replicas ($\Replica{\Cluster{X}}{1}$, $\Replica{\Cluster{X}}{2}$, $\Replica{\Cluster{X}}{3}$, and $\Replica{\Cluster{X}}{4}$), 
and 
cluster $\Cluster{Y}$ has $4$ replicas ($\Replica{\Cluster{Y}}{1}$, $\Replica{\Cluster{Y}}{2}$, $\Replica{\Cluster{Y}}{3}$, and $\Replica{\Cluster{Y}}{4}$).
Say the replicas of $\Cluster{X}$ plan to send a series of ordered messages, starting with sequence number $1$. 
For $\Replica{\Cluster{X}}{1}$: $\Sbuf{1,5,9,...}$ and $\Hbuf{2,3,4,6,7,8,...}$. 
For $\Replica{\Cluster{X}}{2}$: $\Sbuf{2,6,10,...}$ and $\Hbuf{1,3,4,5,7,8,...}$.
For $\Replica{\Cluster{X}}{3}$: $\Sbuf{3,7,11,...}$ and $\Hbuf{1,2,4,5,6,8,...}$.
\end{example}




\begin{itemize}[wide]
\item {\bf Transmit.}
When the $l$-th replica $\Replica{i}{l}$ of the $i$-th \RSM{} $\SMR{i}$ has 
at least one message in its $\Sbuf{l}$, 
it extracts the message with least sequence number; 
say message $\SignMessage{m, k}{\Qusign{i}}$ where no other message in $\Sbuf{l}$ has a sequence number larger than $\Seqn$.
Next, replica $\Replica{i}{l}$ searches $\SAck{l}$ to search for the longest contiguous set of acknowledgments starting from $1$. 
Assume $\Ack{1}, \Ack{2}, ... , \Ack{k'}$ is one such set.
If this is the case, then $\Replica{i}{l}$ appends acknowledgment $\Ack{k'}$ to message $\SignMessage{m, k}{\Qusign{i}}$ and 
sends this to a replica in the $j$-th \RSM{} $\SMR{j}$.


\item {\bf Cumulative Acknowledgments.}
To conserve bandwidth, instead of sending an acknowledgment for each received message, 
\Scrooge{} creates and sends cumulative acknowledgments. 
This requires each replica to search in its list $\SAck{l}$ for the longest contiguous set of acknowledgments. 
The last acknowledgment of this contiguous list is considered as the cumulative acknowledgment for all the 
messages in this list. 

%We require each replica to send a cumulative acknowledgment with every outgoing message 
%even if it implies resending the same acknowledgment (in the case no new messages have been received).

\item {\bf Pairwise Round-Robin.}
Our \Scrooge{} protocol requires each replica $\Replica{i}{l}$ of \RSM{} $\SMR{i}$ to send the message
$\SignMessage{m, k}{\Qusign{i}}$ and acknowledgment $\Ack{k'}$ to exactly one replica of the \RSM{} $\SMR{j}$.
To identify the correct destination, we design a pairwise round-robin algorithm; 
$\Replica{i}{l}$ sends the message to replica $\Replica{j}{r}$ of \RSM{} $\SMR{j}$, 
where $r$ represents the communication round and $r = (l+r) \bmod \n{j} ~\wedge ~r \in [1,\n{j}]$.
Specifically, in each communication round, each replica $\Replica{i}{l}$ 
sends a message to a replica of the other \RSM{}; the destination replica varies per round.
A key advantage of switching destination is to prevent honest replicas to be stuck with byzantine replicas.


%Note: $i$ and $j$ may have different values.
%We do this because $\Replica{\Cluster{X}}{i}$ loops over the set of replicas in cluster $\Cluster{Y}$ and sends 
%distinct messages to each of them.
%In the case of Example~\ref{ex:rep-send}, 
%$\Replica{\Cluster{X}}{1}$ sends the message with sequence $1$ to $\Replica{\Cluster{Y}}{1}$, $5$ to $\Replica{\Cluster{Y}}{2}$, 
%$9$ to $\Replica{\Cluster{Y}}{3}$, $13$ to $\Replica{\Cluster{Y}}{4}$, $17$ to $\Replica{\Cluster{Y}}{1}$, and so on. 


\item {\bf \em Receipt.}
When the $r$-th replica $\Replica{j}{r}$ of the $j$-th \RSM{} receives the 
message $\SignMessage{m, k}{\Qusign{i}}$ and acknowledgment $\Ack{k'}$ from replica $\Replica{i}{l}$ of \RSM{} $\SMR{i}$,
it broadcasts the message and acknowledgment to all the replicas in its \RSM{}.
Next, replica $\Replica{j}{r}$ updates the data-structure $\RAck{r}{}$ as follows:
for all the values $u \in [1,k']$, increment $\RAck{r}{u}$ by one.
If any $\RAck{r}{u}$ reaches the count $\f{j}+1$, it marks the message $\SignMessage{m, u}{\Qusign{j}}$ 
as {\em delivered}.
Further, we keep track of the last message acknowledged by a quorum of replicas (one with highest sequence number till now) in the variable $\highest{}$.

%Say the sequence number for $m$ be $l$.
%If $\Replica{\Cluster{X}}{i}$ has received all the messages with sequence number less than $l$ from cluster $\Cluster{Y}$, 
%it goes ahead and processes $m$. 
%If this is not the case, then $\Replica{\Cluster{X}}{i}$ stores $m$ at the index $l$ in $\Rbuf{}$.
%This $\Rbuf{}$ helps $\Replica{\Cluster{X}}{i}$ to determine the contiguous sequence of messages it has received to send a cumulative 
%acknowledgment in future.
%
%Each message $m'$ received by $\Replica{\Cluster{X}}{i}$ also includes a cumulative acknowledgment ($\Ack{k}$). 
%This is the acknowledgment from the replica $\Replica{\Cluster{Y}}{j}$ (sender of $m'$).
%Through this $\Ack{k}$, $\Replica{\Cluster{Y}}{j}$ wants to inform that it has received all the messages till sequence number $k$ 
%from cluster $\Cluster{X}$.
%This allows $\Replica{\Cluster{X}}{i}$ to increment the {\em counter} for each message with sequence number $\le k$ 
%in its $\Sbuf{}$ and $\Hbuf{}$.
%When the counter for any message reaches $\f{\Replicas{}}+1$, the corresponding message is removed from $\Sbuf{}$ or $\Hbuf{}$.
%We call this message to be {\em majority acknowledged}.


\end{itemize}


%\Scrooge{} requires each honest replica to acknowledge the receipt of a valid message. 
%When a replica $\Replica{i}{l}$ in \RSM{} $\SMR{i}$ receives a message with sequence number $\Seqn$ 
%from the \RSM{} $\SMR{j}$, 
%it creates an acknowledgment $\Ack{k}$ in set $\SAck{l}$.
%
%
%We visualize $\Rbuf{}$ as an array, which stores a copy of received message from the other cluster, 
%indexed by the sequence number.
%We also require replicas to acknowledge (\Name{Ack}) messages they have received from the other cluster. 
%But, sending $\n{}$ \Name{Ack} messages for each received messaged is expensive, so we make replicas send 
%cumulative acknowledgment messages, which help to acknowledge receipt of multiple messages in a shot. 
%In this draft, when a replica sends $\Ack{k}$, it implies that this replica has received all the messages 
%with sequence number less than or equal to $k$.
%Next, we discuss the protocol steps.



{\bf Complexity.}
In the normal-case, when there are no failures, every replica in either \RSM{} will send and receive each message.
Each receiver will get a message either from a replica in the other \RSM{} or through broadcast from a replica in its \RSM{}. 
So, {\em one} message delay to send a message across \RSM{}, and {\em two} message delays to send a message to all the replicas in a \RSM{}.
However, the sender cannot remove a message from its $\Sbuf{}$ or $\Hbuf{}$ unless it receives $\f{}+1$ cumulative $\Ack{}$ messages (majority acknowledged) 
from distinct replicas.
As cumulative acknowledgments imply  receipt of a sequence of messages, so at steady state {\em two} communications across \RSM{} 
per message.






