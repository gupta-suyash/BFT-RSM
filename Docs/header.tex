\usepackage{amsmath,balance}
\usepackage[inline]{enumitem}
\usepackage[binary-units]{siunitx}
\DeclareSIUnit{\nothing}{\relax}
\hyphenation{block-chain}

%% 
%\let\proof\relax
%\let\endproof\relax


%% SI units and transaction notation.
\DeclareSIUnit{\txn}{txn}
\DeclareSIUnit{\batch}{batch}
\sisetup{per-mode=symbol}

%% BFT Protocol names (and Fabric Name).
\newcommand{\Name}[1]{\textsc{#1}}
\newcommand{\BFT}{\textsc{Bft}}
\newcommand{\CFT}{\textsc{Cft}}
\newcommand{\IBC}{\textsc{Ibc}}
\newcommand{\RSM}{\textsc{RSM}}
\newcommand{\PoS}{\textsc{PoS}}
\newcommand{\pbft}{\textsc{Pbft}}
\newcommand{\Shadow}{\textsc{Scrooge}}
\newcommand{\Scrooge}{\textsc{Scrooge}}
\newcommand{\quack}{\textsc{Quack}}
\newcommand{\duck}{\textsc{Duck}}
\newcommand{\Cluster}[1]{\mathcal{#1}}
\newcommand{\Replicas}[1]{\mathcal{R}_{\mathcal{#1}}}
\newcommand{\DS}{\textsc{DS}}

\newcommand{\Replica}[2]{\textsc{R}_{{#1}{#2}}}
\newcommand{\SMR}[1]{\textsc{S}_{#1}}
\newcommand{\CFS}{\textsc{cfs}}
\newcommand{\SWS}{\textsc{sws}}
\newcommand{\RBT}{\textsc{rbt}}
\newcommand{\BST}{\textsc{bst}}
\newcommand{\CCC}{\textsc{c3b}}
\newcommand{\Transmit}{\textsc{transmit}}
\newcommand{\CCCFull}{Cross-Consensus Consistent Broadcast}
\newcommand{\Ped}{$\longrightarrow$ {\em Pedantic Details:}}

\newcommand{\n}[1]{\textbf{n}_{#1}}
\newcommand{\f}[1]{\textbf{f}_{#1}}
\newcommand{\ts}[1]{\textbf{t}_{#1}}
\newcommand{\share}[1]{{\delta}_{#1}}
\newcommand{\Ack}[1]{\Name{Ack}(#1)}
\newcommand{\Packet}[2]{\langle {#1}, \Ack{#2}\rangle}
\newcommand{\Message}[3]{{#1},{#2},\Ack{#3}}
\newcommand{\SignMessage}[2]{\langle#1\rangle_{#2}}
\newcommand{\Seqn}{k}
\newcommand{\Qusign}[1]{\mathcal{Q}_{#1}}
\newcommand{\SList}{\mathcal{L}}
\newcommand{\DList}{\mathcal{D}}

\newcommand{\abs}[1]{\lvert #1 \rvert}
\newcommand{\SID}[1]{\textsc{Sid}({#1})}
\newcommand{\SSet}[1]{\mathcal{M}(#1)}
\newcommand{\RSet}{\mathcal{R}}
\newcommand{\RAS}{\tilde{\mathcal{R}}}
\newcommand{\Sbuf}[1]{\textsc{SBuf}[{#1}]}
\newcommand{\Hbuf}[1]{\textsc{HBuf}[{#1}]}
\newcommand{\Rbuf}[1]{\textsc{RBuf}[{#1}]}
\newcommand{\SAck}[1]{\textsc{SAck}[{#1}]}
\newcommand{\RAck}[2]{\textsc{RAck}[{#1}][{#2}]}
\newcommand{\highest}{\texttt{highest}}


\usepackage{amsthm}
\newtheorem{theorem}{Theorem}[section]
\theoremstyle{definition}
\newtheorem{definition}[theorem]{Definition}
\newtheorem{example}[theorem]{Example}


%% myprotocol
\usepackage[noend]{algorithmic}
\newenvironment{myprotocol}{
    \hrule
    \smallskip
    \scriptsize
    \algsetup{linenosize=\tiny}
    \begin{algorithmic}[1]
        \newcommand{\SPACE}{\item[]}	
	\newcommand{\GETS}{:=}
        \newcommand{\TITLE}[2]{\item[] \textbf{\underline{##1}} (##2) \textbf{:}\\[0.5pt]}
        \makeatletter
            \newcommand{\EVENT}[1]{\STATE \textbf{event} ##1 \textbf{do}\begin{ALC@g}}
            \newcommand{\ENDEVENT}{\end{ALC@g}}
        \makeatother
	
	\makeatletter
            \newcommand{\FUNC}[2]{\STATE \textbf{function} \textbf{##1} (##2) \begin{ALC@g}}
            \newcommand{\ENDFUNC}{\end{ALC@g}}
        \makeatother

	\newcommand{\INITIAL}[2]{\item[] \textbf{\underline{##1}} ##2\\[0.5pt]}
	\newcommand{\MC}[1]{{\color{colN}// ##1}}
}{
    \end{algorithmic}
    \smallskip
    \hrule
}
